% Options for packages loaded elsewhere
\PassOptionsToPackage{unicode}{hyperref}
\PassOptionsToPackage{hyphens}{url}
%
\documentclass[12pt]{article}
\usepackage[top=2.2cm, left=2.2cm, bottom=1.6cm, right=1.8cm]{geometry} % Custom margins

\usepackage{pdfpages}
\usepackage{appendix}

\usepackage{listings}
\usepackage{xcolor}

\lstset{
	language=C++,  % Set the language
	basicstyle=\ttfamily\small,  % Font style
	breaklines=true,  % Automatic line breaking
	breakatwhitespace=false,  % Break lines at whitespace
	postbreak=\mbox{\textcolor{red}{$\hookrightarrow$}\space},  % Mark where text has been broken
	captionpos=b,  % Caption at the bottom
	numbers=left,  % Line numbers on the left
	numberstyle=\tiny\color{gray},  % Styling for the line numbers
	numbersep=5pt,  % Reduced space between line numbers and code
	xleftmargin=5pt,  % Margin on the left outside the frame
	frame=none,  % No frame around the code
	framesep=0pt,  % No padding between frame and code, irrelevant now without frame
	showstringspaces=false,  % Don't show spaces in strings with special character
	tabsize=4,  % Size of tabs
	% linewidth=\textwidth  % Removed as this is not a valid option
}

\usepackage{amssymb,amsmath}
\usepackage{ifxetex,ifluatex}
\ifnum 0\ifxetex 1\fi\ifluatex 1\fi=0 % if pdftex
  \usepackage[T1]{fontenc}
  \usepackage[utf8]{inputenc}
  \usepackage{textcomp} % provide euro and other symbols
\else % if luatex or xetex
  \usepackage{unicode-math}
  \defaultfontfeatures{Scale=MatchLowercase}
  \defaultfontfeatures[\rmfamily]{Ligatures=TeX,Scale=1}
  \usepackage{fontspec}
  \setmainfont{Carlito}
\fi
% Use upquote if available, for straight quotes in verbatim environments
\IfFileExists{upquote.sty}{\usepackage{upquote}}{}
\IfFileExists{microtype.sty}{% use microtype if available
  \usepackage[]{microtype}
  \UseMicrotypeSet[protrusion]{basicmath} % disable protrusion for tt fonts
}{}
\makeatletter
\@ifundefined{KOMAClassName}{% if non-KOMA class
  \IfFileExists{parskip.sty}{%
    \usepackage{parskip}
  }{% else
    \setlength{\parindent}{0pt}
    \setlength{\parskip}{6pt plus 2pt minus 1pt}}
}{% if KOMA class
  \KOMAoptions{parskip=half}}
\makeatother
\usepackage{xcolor}
\IfFileExists{xurl.sty}{\usepackage{xurl}}{} % add URL line breaks if available
\IfFileExists{bookmark.sty}{\usepackage{bookmark}}{\usepackage{hyperref}}
\hypersetup{
  hidelinks,
  pdfcreator={LaTeX via pandoc}}
\urlstyle{same} % disable monospaced font for URLs
\usepackage{longtable,booktabs}
% Correct order of tables after \paragraph or \subparagraph
\usepackage{etoolbox}
\makeatletter
\patchcmd\longtable{\par}{\if@noskipsec\mbox{}\fi\par}{}{}
\makeatother
% Allow footnotes in longtable head/foot
\IfFileExists{footnotehyper.sty}{\usepackage{footnotehyper}}{\usepackage{footnote}}
\makesavenoteenv{longtable}
\usepackage{graphicx}
\makeatletter
\def\maxwidth{\ifdim\Gin@nat@width>\linewidth\linewidth\else\Gin@nat@width\fi}
\def\maxheight{\ifdim\Gin@nat@height>\textheight\textheight\else\Gin@nat@height\fi}
\makeatother
% Scale images if necessary, so that they will not overflow the page
% margins by default, and it is still possible to overwrite the defaults
% using explicit options in \includegraphics[width, height, ...]{}
\setkeys{Gin}{width=\maxwidth,height=\maxheight,keepaspectratio}
% Set default figure placement to htbp
\makeatletter
\def\fps@figure{htbp}
\makeatother
\setlength{\emergencystretch}{3em} % prevent overfull lines
\providecommand{\tightlist}{%
  \setlength{\itemsep}{0pt}\setlength{\parskip}{0pt}}
\setcounter{secnumdepth}{-\maxdimen} % remove section numbering

\title{Individual Project

Final Report}
\author{}
\date{}

\begin{document}
\maketitle

ID Number: B820928

Programme: Electronic and Electrical Engineering MEng

Module Code: 23WSD030/23WS50META

Project Title: Thread by Thread: Benchmarking and Comparing the Executions of Multi-threaded Applications on Multi-core Architectures.

Abstract:
Your abstract here

\newpage
\tableofcontents
\newpage

\section{Introduction(3-5 pages)}
The objective of this project is to enhance the performance of specific applications through parallel computing and to examine their efficiency on both desktop and embedded central processing units (CPUs). Parallel computing is a methodology employed in software development that enables certain segments of code to execute simultaneously, thereby diminishing the overall execution time of an application and augmenting its scalability. Contemporary processors boast a multi-core architecture, integrating multiple CPUs within a single chip, which is pivotal for facilitating parallelism.  

Parallelism can be achieved via multi-threading and multi-processing techniques. A process refers to an active instance of a computer program, encompassing the program's code and its ongoing activities. Each process operates within its distinct memory address space and system resources, maintaining independence from other processes. However, processes can engage in communication with one another through various inter-process communication (IPC) mechanisms. Conversely, a thread represents the minimal unit of processing within a process. It executes within a process's context, sharing the process’s resources like memory and file descriptors, yet it retains its distinct execution state, including the program counter, register set, and stack. Threads can facilitate the parallel execution of code segments within a single process. 

Multi-processing involves an application initiating multiple processes to perform parallel operations, whereas multi-threading pertains to an application that generates multiple threads usually within a single process. This project concentrates on three main applications: the initial two are optimised utilizing multi-threading and their outcomes are assessed on both desktop and embedded processors. The embedded processors examined in this study include the Raspberry Pi 5, 4, and 3, with the Raspberry Pi 5 representing the forefront of technology as of 2024. The final application is a graphical user interface (GUI) application, where multi-processing is employed to attain parallelism. This application was developed by Alex Henneman, a Ph.D. student, under the supervision of Dr. Luciano Ost. The aim here is to refine this application to minimize its execution time and enhance its scalability. 

The inaugural objective of the main project focuses on the enhancement of a multi-core benchmark application named \texttt{mpbenchmark}. The essence of \texttt{mpbenchmark} lies in its capability to simulate real-time calculations of jet engine thermodynamics, serving as a multi-threaded tool to evaluate the multi-core performance of CPUs. This application represents an advanced iteration of its predecessor, \texttt{JetBench}. Initially crafted in \texttt{C} utilizing the \texttt{OpenMP} library, \texttt{JetBench} exhibited certain limitations inherent to its design and programming language. The evolution from \texttt{JetBench} to \texttt{mpbenchmark} marked a significant enhancement, rectifying the previous version's shortcomings. Moreover, \texttt{mpbenchmark} expanded its foundation by incorporating additional programming languages, including \texttt{Ada}, \texttt{C\#}, and \texttt{Java}. Despite this diversification, the implementations in \texttt{C} and \texttt{Ada} were identified as the most performant. However, it's noteworthy that the \texttt{C} and \texttt{Ada} versions of \texttt{mpbenchmark} still encounter language-specific constraints. Specifically, \texttt{C}'s lack of object-oriented capabilities, the requirement for manual memory management, and the absence of libraries for frequently utilized data structures and algorithms pose significant drawbacks. \texttt{Ada}, while offering some advantages, remains a specialized language predominantly employed within the defence and aerospace sectors. To transcend these limitations, this project aspires to develop a modernized version of \texttt{mpbenchmark} utilizing \texttt{C++}. This choice is motivated by \texttt{C++}'s ability to mitigate many of the challenges inherent to \texttt{C}, while still delivering comparable performance. The advent of \texttt{modern C++}, propelled by substantial updates to the core language post-2011, offers promising enhancements that are expected to refine \texttt{mpbenchmark}'s efficiency and functionality.

The secondary objective of this project centres on enhancing the performance of \texttt{MobileNet}, a specialised machine learning algorithm within the realm of convolutional neural networks (\texttt{CNNs}). \texttt{MobileNet} excels in the field of image classification, a process in which the algorithm is provided with an image and tasked with determining the image's corresponding class. While the intricate mathematical foundations and specific operational details of \texttt{MobileNet} are beyond the scope of this discussion, the primary focus is set on augmenting its efficiency. Originally designed as a sequential algorithm, \texttt{MobileNet} primarily operates on a single CPU core, which poses limitations on its execution speed and scalability. In pursuit of addressing these constraints, this project endeavours to transform the existing \texttt{C++} implementation of \texttt{MobileNet} into a multi-threaded architecture. By doing so, the aim is to significantly decrease its runtime and enhance its scalability, thereby optimizing the algorithm's performance for more robust and efficient image classification tasks. This effort to transition \texttt{MobileNet} from a sequential to a multi-threaded model represents a strategic move to leverage the capabilities of modern multi-core processors, ensuring a more responsive and efficient computational process.

The project's third objective is dedicated to the optimisation of a graphical user interface (\texttt{GUI}) application designed for microcontroller (\texttt{MCU}) testing, with a focus on minimising application runtime and bolstering its scalability. Named the \texttt{DeBaTE-FI} platform, this application is crafted in \texttt{Python} and leverages both multi-threading and multi-processing techniques to achieve parallelism and concurrency. It's crucial to distinguish between concurrency and parallelism; in a concurrent framework, tasks are initiated, executed, and completed in overlapping intervals, without necessitating simultaneous operation. The operational flow of the \texttt{DeBaTE-FI} platform involves the user connecting \texttt{MCUs} to a computer, after which the application dispatches specific commands to these \texttt{MCUs}. Following the transmission of commands, the platform awaits responses from the \texttt{MCUs} and proceeds to log the requisite data into output files. Among the three applications addressed in this project, the \texttt{DeBaTE-FI} platform emerges as the most intricate, a complexity attributable to both the scale and the sophisticated nature of its functions. This optimisation effort aims to streamline these extensive operations, enhancing the application's performance and user experience.

The structure of the report is organized as follows: it starts with a literature review, followed by methodology, results, and discussion sections. Appendices are provided after the conclusion for additional reference. The literature review section places the research within the existing body of knowledge by reviewing and interpreting relevant studies pertinent to the research question. This establishes a foundation for understanding the contributions of the project to the field. The methodology section outlines the strategies used to identify performance bottlenecks and potential optimization points within the software, including the introduction of new software designs. These are primarily illustrated through Unified Modeling Language (UML) diagrams and selected code snippets, offering a clear depiction of the proposed solutions. Following the methodology, the results and discussion sections report and analyse the findings from experiments conducted on both desktop and embedded processors, specifically targeting Raspberry Pi models 3 through 5. The findings are presented through benchmark runtime and speedup plots, providing a quantitative evaluation of the performance improvements. In the discussion section, the implications of these results are explored, evaluating the efficiency of the implemented optimizations and proposing directions for future research. This section seeks to provide an in-depth analysis that links the empirical findings with broader theoretical considerations. The appendices serve as a repository for supplementary materials, including details on the experimental PC setup, system specifications, and extended code snippets. These resources offer readers a deeper understanding of the technical aspects of the research, supporting the report's goals of transparency and reproducibility. 

\newpage
\section{Literature review(6-10 pages)}
\subsection{Introduction}
The research topic focuses on parallel computing to optimize algorithms for both desktop and embedded CPUs. Parallel computing is widely regarded as a complex area within engineering and computer science. To develop or enhance multi-threaded solutions, a thorough understanding of the underlying algorithms and the specific goals of the application is crucial. This foundational knowledge is essential before attempting any modifications to ensure that changes do not deviate from the application’s original objectives. In our project, we concentrate on three applications: \texttt{mpbenchmark}, \texttt{MobileNet}, and \texttt{Debate-FI} platform. It is imperative to understand the background and functionality of these applications by utilising the existing publications centred around them. 

Another goal of the literature review is to identify existing solutions to avoid duplicating efforts. Adherence to programming best practices is also a critical aspect of the project, given the complexity of parallel computing. This project strives to ensure that all developed software solutions conform to the best practices and guidelines recommended by industry experts, necessitating a comprehensive literature review.

The literature review is structured to support the report's overall organization, with individual subsections dedicated to the three main objectives, the review covers providing background information, reviewing current research, and offering critical analysis. The review then explores build systems, focusing on how this project aims to develop cross-platform applications that can operate on different operating systems such as \texttt{Linux}, \texttt{Windows}, or \texttt{macOS}. Further research is directed towards software design and multi-threading in line with modern \texttt{C++} guidelines, including discussions on relevant libraries, tools, and research on parallel algorithms. The conclusion section summarizes the findings from the preceding sections and underscores the significance of the reviewed literature to the research question.

\subsection{Objective 1: \texttt{mpbenchmark}}
% Brief overview and relevance to the project 

Our initial focus is on the original \texttt{JetBench} software, which was featured in a publication arising from an international conference held annually in Germany. This conference, part of the ARCS (Architecture of Computing Systems) series, boasts a tradition spanning over 30 years and is renowned for presenting cutting-edge research in computer architecture and operating systems. The specific conference relevant to the first objective of this project took place in 2010. Following the conference, a book was published containing a compilation of papers presented at the event. Our project specifically examines the paper titled ``JetBench: An Open Source Real-Time Multiprocessor Benchmark", authored by Muhammad Yasir Qadri, Dorian Matichard, and Klaus D. McDonald-Maier. This paper is of particular interest to our research as it introduces a multiprocessor benchmark software that aligns well with our project's goals, providing a foundational tool for assessing real-time multiprocessor performance\cite{JetBench_paper}.

To address the identified limitations of the \texttt{JetBench} software, a novel solution was devised using diverse programming languages. This innovation was detailed in a publication titled ``Using JetBench to Evaluate the Efficiency of Multiprocessor Support for Parallel Processing,'' authored by HaiTao Mei and Andy Wellings. The paper, released in 2014, was part of the proceedings of a notable conference\cite{mpbenchmark_paper}.

% Discussion of methodologies used in the studies.
In their research, the authors of \texttt{mpbenchmark}\cite{mpbenchmark_paper} adopted an object-oriented programming (OOP) approach for C\# and Java implementations. They standardized result collection by executing the application 30 times with a varying number of threads, employing a specific Linux command to control CPU core usage. Results were collected using a desktop CPU and \texttt{Simics}(a virtual platform to simulate a high end 128-core CPU).  This methodology is in line with standard practices for benchmark data collection.

% Summary of major findings related to the theme.
The \texttt{JetBench} software emerged in response to the observed scarcity of real-time, multi-threaded benchmarks. Designed to simulate the thermodynamic calculations of jet engines in real time, \texttt{JetBench} serves as an application benchmark that not only simulates a realistic workload but also measures the time required to complete these calculations with different thread counts. Its primary objective is to assess the multi-core capabilities of CPUs. The workload generated by \texttt{JetBench} predominantly consists of Arithmetic Logic Unit (ALU) centric operations, including integer and double precision multiplication, addition, and division. These operations facilitate the computation of critical mathematical functions such as exponentiation, square roots, and conversions including the calculation of pi and degree-to-radian conversions. The software exhibits a significant parallel portion, constituting 88.6\% of its total operations. \texttt{JetBench} is developed in the C programming language and utilizes the \texttt{OpenMP} library to enable multi-threading, ensuring both efficiency and scalability. Furthermore, \texttt{JetBench} prioritizes portability, allowing it to assess the performance of a wide spectrum of systems, from low-end to high-end. This design philosophy ensures the avoidance of system-specific libraries or timers, underlining the software’s broad applicability and utility in performance evaluation across diverse computing environments.

The authors of \texttt{mpbenchmark}\cite{mpbenchmark_paper} highlighted several design flaws in the original \texttt{JetBench} software\cite{JetBench_paper}, including the excessive use of shared variables leading to inaccurate performance results, race conditions from variables shared across threads, and erroneous benchmark output data printing. To remedy these issues, they restructured the \texttt{JetBench} code and introduced implementations in Ada, C\#, and Java. Notably, the compiled languages (C and Ada) demonstrated superior performance. Additionally, the impact of virtual cores enabled by simultaneous multi-threading (SMT) was observed to vary inconsistently across different programming languages.

% Critical evaluation of the strengths, weaknesses, and gaps.
The first paper\cite{JetBench_paper} provides valuable insights into the foundational aspects of this application benchmark, though it is not without its limitations. For instance, the application encompasses a limited scope of computations, a decision made to ensure compatibility with lower-end CPUs. However, this constraint may render the application seemingly inadequate for testing on more advanced CPUs. Additionally, the omission of input/output operations was a deliberate choice to enhance the application's portability, albeit at the expense of being unable to assess the I/O capabilities of the target platform. Despite these limitations, the benchmark's strength lies in its portability and its efficacy in evaluating the multi-core performance across a spectrum of computing systems, from low to high-end. These attributes deem the application an apt selection for this project, which seeks to explore and develop multi-threading capabilities for both desktop and embedded platforms.

In the second paper\cite{mpbenchmark_paper} several limitations were found. Firstly, the data collection, based on just 30 runs, may be insufficient, especially for applications with minimal execution times on high-end CPUs. Additionally, the research focused solely on desktop CPUs and a system simulator mimicking a high-end 128-core CPU, excluding experiments on embedded or lower-end CPUs. Furthermore, given the superior performance of compiled languages like C and Ada, further research could beneficially explore comparisons with \texttt{C++}, another compiled language often regarded as superior to both C and Ada in certain contexts. This paper lays a solid foundation for our project, where its limitations present opportunities for further investigation, and its methodological approaches offer a model for emulation.

\subsection{Objective 2: \texttt{MobileNet}}
% Brief overview and relevance to the project 
MobileNet is a specific type of machine learning algorithm which falls under the convolutional neural networks(CNNs) category. MobileNet was developed to target mobile and embedded vision applications. Since this application is part of our project, the background information regarding this algorithm can be found in a paper published in 2017 by Cornell University titled ``MobileNets: Efficient Convolutional Neural Networks for Mobile Vision Applications"\cite{mobilenet_paper}. 

% Discussion of methodologies used in the studies.
This paper covers the mathematical details that were required to build MobileNet, these details are beyond the scope of this project. When compared with existing CNN models MobileNet results show it to be faster and smaller in size compare to standard CNNs. The paper does not highlight any clear limitations however it does not mention parallel computing techniques used, therefore this presents an opportunity to explore its source code and utilise parallelism to improve performance. 

% Summary of major findings related to the theme.


% Critical evaluation of the strengths, weaknesses, and gaps.
To optimise or enhance MobileNet a \texttt{C++} implementation of MobileNet was selected to be the focus of our project. This was found in a public repository from \texttt{GitHub} website.

The MobileNet implementation was in \texttt{C++} which was used as part of a computer science competition\cite{mobilenet_competition}. 


1- Talk about the mobilenet paper.
2- Talk about the MobileNet github repository and why the C++ implementation was chosen. 


\subsection{Objective 3: \texttt{Debate-FI platform}}

\subsection{Cross platform build systems}

\subsection{\texttt{C++} guidelines and best practices}

\subsection{Literature review conclusion}
Summarize the main findings and debates covered in the review.
Reiterate the importance of the existing literature to your research question.
Outline how your research is positioned within the academic field based on your review.




\newpage
\section{Methodology(5-7 pages)}
\section{Objective 1: \texttt{mpbenchmark}}
As mentioned in the introduction section \texttt{mpbenchmark} performs calculations of a jet engine using multiple threads and produces the time taken to complete these calculations as output. After analysing the code implementation of \texttt{mpbenchmark} the application can be summarised performing the following tasks:

\begin{enumerate}
	\item The application reads data from an input(\texttt{.txt}) file and stores that into an array before starting the calculations. This input data is required to perform calculations required in the next step. 
	\item During calculation, each thread reads input data at specific positions of the input array. After calculation, the results are written into an output array.
	\item In the last step, the benchmark’s response time along with deadlines missed is printed out and saved to an output(\texttt{.txt}) file. 
\end{enumerate}

This design of \texttt{mpbenchmark} can be visualised in figure \ref*{fig:revised_mpbenchmark_structure}.

\begin{figure}[h] % Positioning preference: here, top, bottom, page
	\centering
	\includegraphics[width=0.5\textwidth, height=10cm]{~/Documents/Part_D_Modules/Individual_Project/Individual_report/figures/revised_mpbenchmark_structure.png} % Adjust the path and width as needed
	\caption{Revised \texttt{mpbenchmark} structure \cite{mpbenchmark_paper}.}
	\label{fig:revised_mpbenchmark_structure} % Use this label to reference the figure
\end{figure}

The source code of \texttt{mpbenchmark} provided a solution in \texttt{C\#}, this served as a useful reference of how figure \ref*{fig:revised_mpbenchmark_structure} would be implemented in code using object-oriented design. Subsequently, the \texttt{C++} object oriented design comprised of three main classes:

\begin{enumerate}
	\item \texttt{FileDataLoader}: the primary function of this class is to load data from the input file and also to allow the user to save output data to the output file.
	\item \texttt{SharedPerformanceData}: this class stores data loaded from the input file into an array and also allows storage of output data into a separate array. But importantly it allows threads to access specific parts of the input data in a thread-safe manner. 
	\item \texttt{Worker}: this class contains functions to perform the important calculations, computations of deadlines missed and output data. This class defines the \texttt{operator ()} which encapsulates the main calculations, this class design is know as a \texttt{Functor}.
\end{enumerate}

The \texttt{C++} object-oriented class design can be visualised using a UML class diagram shown in figure \ref*{fig:mpbenchmark_UML_diagram}.

\begin{figure}[h] % Positioning preference: here, top, bottom, page
	\centering
	\includegraphics[width=1\textwidth, height=60cm]{~/Documents/Part_D_Modules/Individual_Project/Individual_report/figures/mpbenchmark_class.png} % Adjust the path and width as needed
	\caption{UML class diagram of the proposed \texttt{C++} solution.}
	\label{fig:mpbenchmark_UML_diagram} % Use this label to reference the figure
\end{figure}

These classes are used in the following sequence in the proposed solution[insert UML sequence diagram]:

The \texttt{mpbenchmark} implementation in the \texttt{C} language utilized the \texttt{printf} function for output throughout the application. Transitioning this functionality to \texttt{C++} posed a challenge, as \texttt{printf} relies on ``format string based" formatting, whereas \texttt{C++} typically uses ``stream-based" formatting through \texttt{std::cout}. This discrepancy was addressed in the \texttt{C++20} standard with the introduction of \texttt{std::format}. However, oddly enough, \texttt{std::format} was not supported by the \texttt{gcc} compiler version on the target system, despite it supporting the \texttt{C++20} standard\cite{std_format_gcc_compiler_version}. A workaround involved using the \texttt{fmt::print} function from the \texttt{fmt} library, which is noted as the inspiration behind \texttt{std::format}\cite{fmt_printing_library}. Detailed system specifications and a code snippet demonstrating the use of \texttt{fmt::print} are provided in the appendix.

The original \texttt{mpbenchmark} code included a command line argument allowing the user (or developer) to specify the engine type for computations. The application supports three distinct engine types. Importantly, the number of threads the application employs can be modified using the \texttt{taskset} command in \texttt{Linux}, which determines the CPU cores the application can run on. To improve this functionality, a second command line argument was added in the proposed solution. This new parameter enables users to set the number of threads the application should use. If this argument is omitted, the application automatically defaults to the maximum number of threads available on the system. This enhancement is detailed in a code snippet in the appendix.

To compile and link the application, the industry standard \texttt{CMake}\cite{cmake_about} software was used. In the \texttt{CMakeLists.txt}(the file used for building the project), the key aspects were specifying the \texttt{C++20} standard, including the \texttt{fmt} library and compilation with the \texttt{-O2} flag. This optimisation flag was also used by the authors of \texttt{mpbenchmark}\cite{mpbenchmark_paper} therefore it was used in the proposed solution for consistency. 

To further enhance the performance of the \texttt{mpbenchmark}, the \texttt{Valgrind/Callgrind} function profiler tool was deployed to find potential bottlenecks. \texttt{Callgrind} profiling results showed that part of the application where it approximates the value of $\pi$ had the highest self-cost. This code snippet is shown in the listing ~\ref{lst:pi_approximation_1} below:

\begin{lstlisting}[
	caption={Part of the code with the highest self-cost. It approximate $\pi$ using numerical integration.},
	label={lst:pi_approximation_1}
	]
// initialise variables for the pi calculation
const long num_steps = 1000000;
double step = 1.0 / static_cast<double>(num_steps);
double x{}, pi{}, sum{};

// performing numerical integration using the midpoint Riemann sum
for (int i = 0; i < num_steps; i++) {
	x = (i + 0.5) * step;
	sum += 4.0 / (1.0 + x * x);
}
pi = sum * step;
\end{lstlisting}

The problem with this part of the code is that it already part of the parallel regions where it is executed by each thread. It may seen like a good candidate for applying parallel \texttt{for loop} from the OpenMP library however creating nested threads beyond the number of threads available on the system does not always lead to higher performance and in many cases can degrade performance. Another way to improve performance is by using SIMD intrinsics, this a programming tool to improve single-threaded or sequential performance of a code it stands for ``Single Instruction Multiple Data". By using these intrinsics, programmers can write code that processes data in parallel directly within a single CPU cycle. An advantage of using \texttt{C++}(and/or \texttt{C}) is that SIMD intrinsics can be deployed whereas higher level languages like \texttt{Java} or \texttt{C\#} make it very difficult to access these. 

SIMD intrinsics vary by the target CPU, \texttt{x86} processors (which are found in most laptop and desktops) use \texttt{AVX2} instructions whereas \texttt{ARM} processors(commonly found in Apple products and embedded devices) use \texttt{NEON} instructions. In this project we need to use both as our proposed \texttt{C++} solution will be deployed on both desktop(\texttt{x86} processor) and Raspberry Pi devices(\texttt{ARM} based processor). The SIMD enhanced code can be summarised algorithmatically in the following steps and shown as a code snippet in listing ~\ref{lst:pi_approximation_2}: 

\begin{enumerate}
	\item Initialise \texttt{256-bit} wide vectors: each vector can hold four double-precision (\texttt{64-bit}) floating point numbers. The main initialisations would be a vector to hold four loop indices (\texttt{vec\_i}), a vector to calculate four values of $x$ (\texttt{vec\_x}), a vector to hold the result of the integrand (\texttt{vec\_temp}) and a vector to accumulate the sum (\texttt{vec\_sum}) after each loop iteration.
	\item \texttt{for loop} iterate until \texttt{num\_steps/4}:
	\begin{itemize}
		\item step 1: calculate the four midpoints $x$-values simultaneously using the vector \texttt{vec\_i} and hold result in \texttt{vec\_x}. Original formula used: \texttt{(i + 0.5) * step}.
		\item step 2: compute the value of the integrand in parallel using the four calculated $x$-values stored in \texttt{vec\_x}, store result in \texttt{vec\_temp}. Original formula used: \texttt{4 / (1 + x * x)}.
		\item step 3: accumulate the values from \texttt{vec\_temp} into the \texttt{vec\_sum} vector.
		\item step 4: increment loop indices vector \texttt{vec\_i} by \texttt{4}. 
	\end{itemize}
	\item Final summation: upon the completion of the loop, perform a horizontal sum on the vector (\texttt{vec\_sum}) that held the accumulated values.
	\item Calculation of $\pi$: sum is multiplied by the step size to approximate the value of $\pi$. Original formula used : \texttt{pi = sum * step}.
\end{enumerate}

The SIMD enhanced code implemented using \texttt{AVX2} instructions of listing ~\ref{lst:pi_approximation_1} is shown below in listing ~\ref{lst:pi_approximation_2}: 

\begin{lstlisting}[
	caption={SIMD enhanced code for approximation of $\pi$ using \texttt{AVX2} instructions. (Horizontal sum function in line 28 can be found in appendix).},
	label={lst:pi_approximation_2}
	]
double Worker::approximatePi(){
	double pi = 0.0; // Initialize pi to 0.0
	static constexpr long num_steps = 1000000; 
	static constexpr double step = 1.0 / static_cast<double>(num_steps); 
	
#if defined(__AVX2__)
	// Use AVX2 SIMD instructions if available
	double sum = 0; // Initialize scalar sum to accumulate final result
	
// Initialise all necessary 256-bit vectors
	__m256d vec_sum = _mm256_set1_pd(0.0);
	__m256d vec_step = _mm256_set1_pd(step);
	__m256d vec_half_step = _mm256_set1_pd(0.5 * step); 
	__m256d vec_one = _mm256_set1_pd(1.0); 
	__m256d vec_four = _mm256_set1_pd(4.0); 
	__m256d vec_x, vec_temp; 
	__m256d vec_i = _mm256_set_pd(3, 2, 1, 0); 
	__m256d vec_increment = _mm256_set1_pd(4); 

// Perform 4 computations at once, note "i" is incremented by 4 instead of 1
	for (int i = 0; i < num_steps; i += 4) {
		vec_x    = _mm256_add_pd(_mm256_mul_pd(vec_i, vec_step), vec_half_step); 
		vec_temp = _mm256_div_pd(vec_four, _mm256_add_pd(vec_one, _mm256_mul_pd(vec_x, vec_x))); 
		vec_sum  = _mm256_add_pd(vec_sum, vec_temp); 
		vec_i    = _mm256_add_pd(vec_i, vec_increment); 
	}
// Perform horizontal sum on vec_sum to get a scalar sum
	sum = hsum256_pd(vec_sum);
 // Multiply the sum by the step size to approximate pi
	pi = sum * step; 

#else
// If AVX2/SIMD instructions are unavailable then resort to using regular code... 
#endif

	return pi; 
}
\end{lstlisting}

As discussed, to utilize SIMD intrinsics on Raspberry Pi devices, \texttt{NEON} instructions must be employed. \texttt{NEON} instructions come with limitations, notably in their support for double precision floating points, which is restricted, and their vector width, which is only \texttt{128-bit}, compared to the \texttt{256-bit} vectors seen in \texttt{AVX2}\cite{neon_reference}. To accommodate \texttt{NEON}, two solutions were developed: one using single precision floating points (\texttt{float}) and the other using double precision floating points (\texttt{double}). The \texttt{NEON} code with \texttt{float} can perform four computations simultaneously, while the code with \texttt{double} can manage only two computations simultaneously, due to the \texttt{128-bit} vector's capacity to store four \texttt{float} values or two \texttt{double} values. Typically, \texttt{float} variables offer less decimal precision than \texttt{double} variables. These two SIMD-enhanced solutions, along with their decimal accuracy levels, will be compared in the results and discussion section. The \texttt{NEON} implementation follows the same algorithm as the code snippet shown in Listing~\ref{lst:pi_approximation_2}, and is available in the appendix.

In summary, for desktop (\texttt{x86}) processors, two main solutions have been developed: a novel \texttt{C++} solution and a SIMD-enhanced \texttt{C++} solution as discussed in Listing~\ref{lst:pi_approximation_2}. For the Raspberry Pi devices, three solutions have been created: the first is the novel \texttt{C++} solution, identical to that on the desktop, and the other two are the SIMD-enhanced versions utilizing \texttt{NEON} instructions with single and double precision floating point variables.

\section{Objective 2: \texttt{MobileNet}}
\section{Objective 3: \texttt{DeBaTE-FI platform}}

\section{Results and Discussion(8-10 pages)}

\subsection{Objective 1: \texttt{mpbenchmark}}
\subsection{Objective 2: \texttt{MobileNet}}
\subsection{Objective 3: \texttt{DeBaTE-FI platform}}

\section{Conclusion(2-4 pages)}

\bibliographystyle{IEEEtran}
\bibliography{references}
\addcontentsline{toc}{section}{References}  % Add this line


% Input the appendix file here
\begin{appendices}
	\section{System and Raspberry Pi specifications}
\section{Objective 1: \texttt{mpbenchmark}}
\section{Objective 2: \texttt{MobileNet}}
\section{Objective 3: \texttt{DeBaTE-FI} platform}

\begin{lstlisting}[
	caption={Command line arguments of the new \texttt{mpbenchmark} solution.},
	label={lst:main_file_cpp}
	]
	int main(int argc, char *argv[]){
		int engine{};
		int numThreads{};
		
		if (argc > 1) {
			engine = std::atoi(argv[1]); // Convert the argument to an integer
		}
		
		if (argc > 2) {
			numThreads = std::atoi(argv[2]); // Convert the argument to an integer
		}
		
		// If the number of threads is not specified, default to the maximum available
		if (numThreads == 0) {
			numThreads = std::thread::hardware_concurrency();
		}	
		// ignore the other remaining code...
	}
\end{lstlisting}


\includepdf[
pages=-,
addtotoc={1, section, 1, DeBaTE-FI platform publication(maybe remove?), appendix:debate_fi},
pagecommand={\label{appendix:debate_fi}}
]{~/Documents/Part_D_Modules/Individual_Project/Individual_report/files/Debate_FI_platform.pdf}



\end{appendices}

\end{document}



