% Options for packages loaded elsewhere
\PassOptionsToPackage{hyphens}{url}
%
\documentclass[12pt, openany]{book}
\usepackage[top=2.2cm, left=2.2cm, bottom=1.6cm, right=1.8cm]{geometry} % Custom margins

\usepackage{pdfpages}
\usepackage[titletoc]{appendix}

\usepackage{listings}
\usepackage{xcolor}

\usepackage{float}
\usepackage{array}  % for better column formatting

\usepackage{seqsplit}


\usepackage{cite} % for IEEE citation format

% Adding titlesec package
\usepackage{titlesec}
\usepackage{tabularx} % for better table control

% Redefining the chapter title format
\titleformat{\chapter}[display]
{\normalfont\huge\bfseries}{\chaptertitlename\ \thechapter}{20pt}{\Huge}
\titlespacing*{\chapter}{0pt}{-50pt}{10pt}

% Define colors for syntax highlighting
\definecolor{commentcolor}{gray}{0.5}
\definecolor{keywordcolor}{rgb}{0,0,1}
\definecolor{stringcolor}{rgb}{0.58,0,0.82}

\lstset{
	language=C++,  % Set the language
	basicstyle=\ttfamily\fontsize{10pt}{12pt}\selectfont,  % Set font style and size to 10pt with a normal line spacing of 12pt
	breaklines=true,  % Automatic line breaking
	captionpos=b,  % Caption at the bottom
	numbers=left,  % Line numbers on the left
	numberstyle=\tiny\color{gray},  % Styling for the line numbers
	numbersep=5pt,  % Space between line numbers and code
	xleftmargin=5pt,  % Margin on the left outside the frame
	frame=none,  % No frame around the code
	showstringspaces=false,  % Don't show spaces in strings with special character
	tabsize=4,  % Size of tabs
	commentstyle=\color{commentcolor}\ttfamily,
	keywordstyle=\color{keywordcolor}\bfseries\ttfamily,  % Bold for keywords
	stringstyle=\color{stringcolor}\ttfamily,
	morecomment=[l][\color{magenta}]{\#}
}

\renewcommand{\bibname}{References}

\usepackage{fontspec}
\setmainfont{Carlito}

\usepackage{amssymb,amsmath}
\usepackage{ifxetex,ifluatex}
\ifnum 0\ifxetex 1\fi\ifluatex 1\fi=0 % if pdftex
  \usepackage[T1]{fontenc}
  \usepackage[utf8]{inputenc}
  \usepackage{textcomp} % provide euro and other symbols
\else % if luatex or xetex
  \usepackage{unicode-math}
  \defaultfontfeatures{Scale=MatchLowercase}
  \defaultfontfeatures[\rmfamily]{Ligatures=TeX,Scale=1}
  \usepackage{fontspec}
  \setmainfont{Carlito}
\fi
% Use upquote if available, for straight quotes in verbatim environments
\IfFileExists{upquote.sty}{\usepackage{upquote}}{}
\IfFileExists{microtype.sty}{% use microtype if available
  \usepackage[]{microtype}
  \UseMicrotypeSet[protrusion]{basicmath} % disable protrusion for tt fonts
}{}

\makeatletter
\@ifundefined{KOMAClassName}{% if non-KOMA class
	\IfFileExists{parskip.sty}{%
		\usepackage{parskip}
	}{% else
		\setlength{\parindent}{0pt}
		\setlength{\parskip}{6pt plus 2pt minus 1pt}}
}{% if KOMA class
	\KOMAoptions{parskip=half}}
\makeatother
\usepackage{xcolor}
\IfFileExists{xurl.sty}{\usepackage{xurl}}{} % add URL line breaks if available
\IfFileExists{bookmark.sty}{\usepackage{bookmark}}{\usepackage{hyperref}}
\hypersetup{
	hidelinks,
	pdfcreator={LaTeX via pandoc}}
\urlstyle{same} % disable monospaced font for URLs
\usepackage{longtable,booktabs}

% Correct order of tables after \paragraph or \subparagraph
\usepackage{etoolbox}
\makeatletter
\patchcmd\longtable{\par}{\if@noskipsec\mbox{}\fi\par}{}{}
\makeatother
% Allow footnotes in longtable head/foot
\IfFileExists{footnotehyper.sty}{\usepackage{footnotehyper}}{\usepackage{footnote}}
\makesavenoteenv{longtable}
\usepackage{graphicx}
\makeatletter
\def\maxwidth{\ifdim\Gin@nat@width>\linewidth\linewidth\else\Gin@nat@width\fi}
\def\maxheight{\ifdim\Gin@nat@height>\textheight\textheight\else\Gin@nat@height\fi}
\makeatother
% Scale images if necessary, so that they will not overflow the page
% margins by default, and it is still possible to overwrite the defaults
% using explicit options in \includegraphics[width, height, ...]{}
\setkeys{Gin}{width=\maxwidth,height=\maxheight,keepaspectratio}
% Set default figure placement to htbp
\makeatletter
\def\fps@figure{htbp}
\makeatother
\setlength{\emergencystretch}{3em} % prevent overfull lines
\providecommand{\tightlist}{%
  \setlength{\itemsep}{0pt}\setlength{\parskip}{0pt}}

% Include fancyhdr for header and footer customization
\usepackage{fancyhdr}

% Page style setup
\pagestyle{fancy}
\fancyhf{} % clear all header and footer fields
\fancyfoot[C]{\thepage} % put the page number in the center of the footer
\renewcommand{\headrulewidth}{0pt} % no line in header area
\renewcommand{\footrulewidth}{0pt} % no line in footer area

% Ensure the plain page style, used on chapter opening pages, is also empty
\fancypagestyle{plain}{
	\fancyhf{} % clear all header and footer fields for plain pages
	\fancyfoot[C]{\thepage} % page number in the center of the footer on plain pages
	\renewcommand{\headrulewidth}{0pt}
	\renewcommand{\footrulewidth}{0pt}
}



\usepackage{titlesec}

% Formatting the chapter title to remove "Chapter X"
\titleformat{\chapter}[display]
{\normalfont\huge\bfseries}{}{0pt}{\Huge}


\begin{document}
\begin{titlepage}
	\centering
	% Title
	\vspace*{1cm} % Adjust the vertical space as needed
	{\Huge\bfseries Individual Project\\[0.5cm] Final Report\par}
	\vspace{2cm}
	% Reset centering to left-align the rest
	\raggedright
	% ID Number
	ID Number: B820928\par
	
	% Programme
	Programme: Electronic and Electrical Engineering MEng\par
	
	% Module Code
	Module Code: 23WSD030/23WS50META\par
	
	% Project Title
	Project Title: Thread by Thread: Benchmarking and Comparing the Executions of Multi-threaded Applications on Multi-core Architectures.\par
	
	% Abstract Heading
	\textbf{Abstract:}\par % Making 'Abstract' bold
	% Abstract Text
	% Background, aims, methods, results/conclusions.

	Modern processors, both in desktop and embedded systems, increasingly feature multi-core architectures. This project aimed to optimise three applications using parallel computing techniques to decrease runtime and enhance scalability, testing them on desktop and embedded processors, including the Raspberry Pi 5. The applications included \texttt{mpbenchmark}, a benchmarking tool for multi-core performance; \texttt{MobileNet}, a machine learning application for image classification originally single-threaded and adapted to use the \texttt{OpenMP} library for multi-threading; and \texttt{DeBaTE-FI}, a GUI for testing microcontrollers for soft errors, optimised with Python’s multi-processing and a high-performance C++ library for telnet communication. Results showed significant performance improvements: \texttt{mpbenchmark} was redesigned with modern \texttt{C++} and SIMD intrinsics, improving scalability and design; \texttt{MobileNet} achieved an 87\% performance increase on desktop processors, with a 60-65\% gain on Raspberry Pi; and \texttt{Debate-FI} saw a 43.5-62.1\% reduction in runtime across various setups. The project's contributions include a novel \texttt{C++} benchmark tool, the first parallel version of the \texttt{MobileNet} CNN model, and an optimised \texttt{Debate-FI} platform, all demonstrating enhanced performance and scalability.

\end{titlepage}


\newpage
\tableofcontents

\chapter{Introduction}
%Introduction(3-5 pages)
The objective of this project is to enhance the performance of specific applications through parallel computing and to examine their efficiency on both desktop and embedded central processing units (CPUs). Parallel computing is a methodology employed in software development that enables certain segments of code to execute simultaneously, thereby diminishing the overall execution time of an application and augmenting its scalability. Contemporary processors boast a multi-core architecture, integrating multiple CPUs within a single chip, which is pivotal for facilitating parallelism\cite{modern_processors}.  

Parallelism can be achieved via multi-threading and multi-processing techniques. A process refers to an active instance of a computer program, encompassing the program's code and its ongoing activities. Each process operates within its distinct memory address space and system resources, maintaining independence from other processes. However, processes can engage in communication with one another through various inter-process communication (IPC) mechanisms. Conversely, a thread represents the minimal unit of processing within a process. It executes within a process's context, sharing the process’s resources like memory and file descriptors, yet it retains its distinct execution state, including the program counter, register set, and stack. Threads can facilitate the parallel execution of code segments within a single process\cite{multi_processing_multi_threading_article}.

Multi-processing involves an application initiating multiple processes to perform parallel operations, whereas multi-threading pertains to an application that generates multiple threads usually within a single process. This project concentrates on three main applications: the initial two are optimised utilizing multi-threading and their outcomes are assessed on both desktop and embedded processors. The embedded processors examined in this study include the Raspberry Pi 5, 4, and 3, with the Raspberry Pi 5 representing the forefront of technology as of 2024. The final application is a graphical user interface (GUI) application, where multi-processing is employed to attain parallelism. This application was developed by Alex Henneman, a Ph.D. student, under the supervision of Dr. Luciano Ost\cite{debate_fi_publication}. The aim here is to refine this application to minimize its execution time and enhance its scalability. 

The first objective of this project is to improve the \texttt{mpbenchmark} multi-core benchmark tool, which simulates jet engine thermodynamics to evaluate CPU performance \cite{mpbenchmark_paper}. \texttt{mpbenchmark} generates two key outputs: runtime and the number of deadlines missed, the latter indicating performance quality based on thread completion times. This tool builds upon its predecessor, \texttt{JetBench}, which was originally developed in \texttt{C} using \texttt{OpenMP} \cite{JetBench_paper}. The transition to \texttt{mpbenchmark} addresses \texttt{JetBench}'s limitations and extends support to other languages like \texttt{Ada}, \texttt{C\#}, and \texttt{Java}, with \texttt{C} and \texttt{Ada} showing the best performance. However, these versions face challenges such as \texttt{C}'s lack of object-orientation and manual memory management issues \cite{c_language_drawbacks}, and \texttt{Ada}'s limited use outside defense and aerospace \cite{ada_langauge_uses}. To overcome these, the project aims to develop \texttt{mpbenchmark} using \texttt{C++}, leveraging modern enhancements which were introduced in the language since 2011\cite{evolution_of_C++}.

The secondary objective of this project centres on enhancing the performance of \texttt{MobileNet}, a specialised machine learning algorithm within the realm of convolutional neural networks (\texttt{CNNs})\cite{mobilenet_paper}. \texttt{MobileNet} excels in the field of image classification, a process in which the algorithm is provided with an image and tasked with determining the image's corresponding class. While the intricate mathematical foundations and specific operational details of \texttt{MobileNet} are beyond the scope of this discussion, the primary focus is set on augmenting its efficiency. Originally designed as a sequential algorithm, \texttt{MobileNet} primarily operates on a single CPU core, which poses limitations on its execution speed and scalability. In pursuit of addressing these constraints, this project endeavours to transform the existing \texttt{C++} implementation of \texttt{MobileNet}\cite{mobilenet_repo} into a multi-threaded architecture. By doing so, the aim is to significantly decrease its runtime and enhance its scalability, thereby optimizing the algorithm's performance for more robust and efficient image classification tasks. This effort to transition \texttt{MobileNet} from a sequential to a multi-threaded model represents a strategic move to leverage the capabilities of modern multi-core processors, ensuring a more responsive and efficient computational process.

The project's third objective is dedicated to the optimisation of a graphical user interface (\texttt{GUI}) application designed for microcontroller (\texttt{MCU}) testing, with a focus on minimising application runtime and bolstering its scalability. Named the \texttt{DeBaTE-FI} platform\cite{debate_fi_publication}, this application is crafted in \texttt{Python} and leverages both multi-threading and multi-processing techniques to achieve parallelism and concurrency. It's crucial to distinguish between concurrency and parallelism; in a concurrent framework, tasks are initiated, executed, and completed in overlapping intervals, without necessitating simultaneous operation. The operational flow of the \texttt{DeBaTE-FI} platform involves the user connecting \texttt{MCUs} to a computer, after which the application dispatches specific commands to these \texttt{MCUs}. Following the transmission of commands, the platform awaits responses from the \texttt{MCUs} and proceeds to log the requisite data into output files. Among the three applications addressed in this project, the \texttt{DeBaTE-FI} platform emerges as the most intricate, a complexity attributable to both the scale and the sophisticated nature of its functions. This optimisation effort aims to streamline these extensive operations, enhancing the application's performance and user experience.

The report is structured as follows: It begins with a literature review that contextualises the research within the field by examining relevant studies. This is followed by the methodology section, which details strategies for identifying and addressing performance bottlenecks in the software, illustrated through Unified Modelling Language (UML) diagrams and code snippets. The results and discussion sections then present and analyse the performance enhancements observed in experiments conducted on desktop and Raspberry Pi models 3 through 5, with benchmark runtime and speedup plots quantifying the improvements. The discussion evaluates the optimisations' effectiveness and suggests future research directions, linking practical outcomes with theoretical implications. Appendices follow the conclusion, providing supplementary materials such as experimental setup details, system specifications, and extended code snippets to enhance understanding and ensure the report's transparency and reproducibility.

\newpage
\section{Original contributions of this project}

\begin{enumerate}
	\item A novel multi-core benchmark application has been developed using the latest \texttt{C++} features, this application built upon the previous \texttt{mpbenchmark}\cite{mpbenchmark_paper} application. It outperformed the previous \texttt{mpbenchmark} implementations in \texttt{C} and \texttt{Ada} by 11\%, 4\% and 1\% on desktop, Raspberry Pi 5 and 4 respectively in terms of run time and provided better speedup across threads. Moreover the novel benchmark application also utilised SIMD intrinsics which provided up to an astonishing 70\% reduction in run time, this unprecedented improvement provided means of analysing the target CPU's SIMD operations ability. A publication describing the contribution of this project has been submitted to SBCCI(Symposium on Integrated Circuits and System Design) conference part of IEEE(Institute of Electrical and Electronics Engineers) organisation. 
	
	\item The development of the first multi-threaded \texttt{MobileNet} CNN(convolutional neural network) model. This enhanced CNN model has been used as a case study to assess the behaviour of multi-threaded applications running on multi-core processors under neutron radiation. A publication describing this CNN model has been submitted to IEEE RADECS(``RADiation Effects on Components and Systems") conference.
	
	\item The optimised redesign of the \texttt{DeBaTE-FI} platform\cite{debate_fi_publication} resulted in up to 60\% reduction in run time. Moreover a high performance \texttt{C++} library was developed for \texttt{telnet} communication with microcontrollers. The enhanced \texttt{DeBaTE-FI} platform produced a 43.5\% reduction in run time when testing 36 \texttt{STM32F767ZI} microcontrollers when tested in PhD research office in Loughborough University(Figure ~\ref{fig:debate_fi_setup}).
\end{enumerate}


\chapter{Literature review}
% Literature review(6-10 pages)
\subsection{Introduction}
The research topic focuses on parallel computing to optimize algorithms for both desktop and embedded CPUs. Parallel computing is widely regarded as a complex area within engineering and computer science. To develop or enhance multi-threaded solutions, a thorough understanding of the underlying algorithms and the specific goals of the application is crucial. This foundational knowledge is essential before attempting any modifications to ensure that changes do not deviate from the application’s original objectives. In our project, we concentrate on three applications: \texttt{mpbenchmark}, \texttt{MobileNet}, and \texttt{Debate-FI} platform. It is imperative to understand the background and functionality of these applications by utilising the existing publications centred around them. 

Another goal of the literature review is to identify existing solutions to avoid duplicating efforts. Adherence to programming best practices is also a critical aspect of the project, given the complexity of parallel computing. This project strives to ensure that all developed software solutions conform to the best practices and guidelines recommended by industry experts, necessitating a comprehensive literature review.

The literature review is structured to support the report's overall organization, with individual subsections dedicated to the three main objectives, the review covers providing background information, reviewing current research, and offering critical analysis. The review then explores build systems, focusing on how this project aims to develop cross-platform applications that can operate on different operating systems such as \texttt{Linux}, \texttt{Windows}, or \texttt{macOS}. Further research is directed towards software design and multi-threading in line with modern \texttt{C++} guidelines, including discussions on relevant libraries, tools, and research on parallel algorithms. The conclusion section summarizes the findings from the preceding sections and underscores the significance of the reviewed literature to the research question.

\subsection{Objective 1: \texttt{mpbenchmark}}
% Brief overview and relevance to the project 

Our initial focus is on the original \texttt{JetBench} software, which was featured in a publication arising from an international conference held annually in Germany. This conference, part of the ARCS (Architecture of Computing Systems) series, boasts a tradition spanning over 30 years and is renowned for presenting cutting-edge research in computer architecture and operating systems. The specific conference relevant to the first objective of this project took place in 2010. Following the conference, a book was published containing a compilation of papers presented at the event. Our project specifically examines the paper titled ``JetBench: An Open Source Real-Time Multiprocessor Benchmark", authored by Muhammad Yasir Qadri, Dorian Matichard, and Klaus D. McDonald-Maier. This paper is of particular interest to our research as it introduces a multiprocessor benchmark software that aligns well with our project's goals, providing a foundational tool for assessing real-time multiprocessor performance\cite{JetBench_paper}.

To address the identified limitations of the \texttt{JetBench} software, a novel solution was devised using diverse programming languages. This innovation was detailed in a publication titled ``Using JetBench to Evaluate the Efficiency of Multiprocessor Support for Parallel Processing,'' authored by HaiTao Mei and Andy Wellings. The paper, released in 2014, was part of the proceedings of a notable conference\cite{mpbenchmark_paper}.

% Discussion of methodologies used in the studies.
In their research, the authors of \texttt{mpbenchmark}\cite{mpbenchmark_paper} adopted an object-oriented programming (OOP) approach for C\# and Java implementations. They standardized result collection by executing the application 30 times with a varying number of threads, employing a specific Linux command to control CPU core usage. Results were collected using a desktop CPU and \texttt{Simics}(a virtual platform to simulate a high end 128-core CPU).  This methodology is in line with standard practices for benchmark data collection.

% Summary of major findings related to the theme.
The \texttt{JetBench} software emerged in response to the observed scarcity of real-time, multi-threaded benchmarks. Designed to simulate the thermodynamic calculations of jet engines in real time, \texttt{JetBench} serves as an application benchmark that not only simulates a realistic workload but also measures the time required to complete these calculations with different thread counts. Its primary objective is to assess the multi-core capabilities of CPUs. The workload generated by \texttt{JetBench} predominantly consists of Arithmetic Logic Unit (ALU) centric operations, including integer and double precision multiplication, addition, and division. These operations facilitate the computation of critical mathematical functions such as exponentiation, square roots, and conversions including the calculation of pi and degree-to-radian conversions. The software exhibits a significant parallel portion, constituting 88.6\% of its total operations. \texttt{JetBench} is developed in the C programming language and utilizes the \texttt{OpenMP} library to enable multi-threading, ensuring both efficiency and scalability. Furthermore, \texttt{JetBench} prioritizes portability, allowing it to assess the performance of a wide spectrum of systems, from low-end to high-end. This design philosophy ensures the avoidance of system-specific libraries or timers, underlining the software’s broad applicability and utility in performance evaluation across diverse computing environments.

The authors of \texttt{mpbenchmark}\cite{mpbenchmark_paper} highlighted several design flaws in the original \texttt{JetBench} software\cite{JetBench_paper}, including the excessive use of shared variables leading to inaccurate performance results, race conditions from variables shared across threads, and erroneous benchmark output data printing. To remedy these issues, they restructured the \texttt{JetBench} code and introduced implementations in Ada, C\#, and Java. Notably, the compiled languages (C and Ada) demonstrated superior performance. Additionally, the impact of virtual cores enabled by simultaneous multi-threading (SMT) was observed to vary inconsistently across different programming languages.

% Critical evaluation of the strengths, weaknesses, and gaps.
The first paper\cite{JetBench_paper} provides valuable insights into the foundational aspects of this application benchmark, though it is not without its limitations. For instance, the application encompasses a limited scope of computations, a decision made to ensure compatibility with lower-end CPUs. However, this constraint may render the application seemingly inadequate for testing on more advanced CPUs. Additionally, the omission of input/output operations was a deliberate choice to enhance the application's portability, albeit at the expense of being unable to assess the I/O capabilities of the target platform. Despite these limitations, the benchmark's strength lies in its portability and its efficacy in evaluating the multi-core performance across a spectrum of computing systems, from low to high-end. These attributes deem the application an apt selection for this project, which seeks to explore and develop multi-threading capabilities for both desktop and embedded platforms.

In the second paper\cite{mpbenchmark_paper} several limitations were found. Firstly, the data collection, based on just 30 runs, may be insufficient, especially for applications with minimal execution times on high-end CPUs. Additionally, the research focused solely on desktop CPUs and a system simulator mimicking a high-end 128-core CPU, excluding experiments on embedded or lower-end CPUs. Furthermore, given the superior performance of compiled languages like C and Ada, further research could beneficially explore comparisons with \texttt{C++}, another compiled language often regarded as superior to both C and Ada in certain contexts. This paper lays a solid foundation for our project, where its limitations present opportunities for further investigation, and its methodological approaches offer a model for emulation.

\subsection{Objective 2: \texttt{MobileNet}}
% Brief overview and relevance to the project 
MobileNet is a specific type of machine learning algorithm which falls under the convolutional neural networks(CNNs) category. MobileNet was developed to target mobile and embedded vision applications. Since this application is part of our project, the background information regarding this algorithm can be found in a paper published in 2017 by Cornell University titled ``MobileNets: Efficient Convolutional Neural Networks for Mobile Vision Applications"\cite{mobilenet_paper}. 

% Discussion of methodologies used in the studies.
This paper covers the mathematical details that were required to build MobileNet, these details are beyond the scope of this project. When compared with existing CNN models MobileNet results show it to be faster and smaller in size compare to standard CNNs. The paper does not highlight any clear limitations however it does not mention parallel computing techniques used, therefore this presents an opportunity to explore its source code and utilise parallelism to improve performance. 

% Summary of major findings related to the theme.


% Critical evaluation of the strengths, weaknesses, and gaps.
To optimise or enhance MobileNet a \texttt{C++} implementation of MobileNet was selected to be the focus of our project. This was found in a public repository from \texttt{GitHub} website.

The MobileNet implementation was in \texttt{C++} which was used as part of a computer science competition\cite{mobilenet_competition}. 


1- Talk about the mobilenet paper.
2- Talk about the MobileNet github repository and why the C++ implementation was chosen. 


\subsection{Objective 3: \texttt{Debate-FI platform}}

\subsection{Cross platform build systems}

\subsection{\texttt{C++} guidelines and best practices}

\subsection{Literature review conclusion}
Summarize the main findings and debates covered in the review.
Reiterate the importance of the existing literature to your research question.
Outline how your research is positioned within the academic field based on your review.




\chapter{Methodology}
%Methodology(5-7 pages)
\section{Objective 1: \texttt{mpbenchmark}}
As mentioned in the introduction section \texttt{mpbenchmark} performs calculations of a jet engine using multiple threads and produces the time taken to complete these calculations as output. After analysing the code implementation of \texttt{mpbenchmark} the application can be summarised performing the following tasks:

\begin{enumerate}
	\item The application reads data from an input(\texttt{.txt}) file and stores that into an array before starting the calculations. This input data is required to perform calculations required in the next step. 
	\item During calculation, each thread reads input data at specific positions of the input array. After calculation, the results are written into an output array.
	\item In the last step, the benchmark’s response time along with deadlines missed is printed out and saved to an output(\texttt{.txt}) file. 
\end{enumerate}

This design of \texttt{mpbenchmark} can be visualised in figure \ref*{fig:revised_mpbenchmark_structure}.

\begin{figure}[h] % Positioning preference: here, top, bottom, page
	\centering
	\includegraphics[width=0.5\textwidth, height=10cm]{~/Documents/Part_D_Modules/Individual_Project/Individual_report/figures/revised_mpbenchmark_structure.png} % Adjust the path and width as needed
	\caption{Revised \texttt{mpbenchmark} structure \cite{mpbenchmark_paper}.}
	\label{fig:revised_mpbenchmark_structure} % Use this label to reference the figure
\end{figure}

The source code of \texttt{mpbenchmark} provided a solution in \texttt{C\#}, this served as a useful reference of how figure \ref*{fig:revised_mpbenchmark_structure} would be implemented in code using object-oriented design. Subsequently, the \texttt{C++} object oriented design comprised of three main classes:

\begin{enumerate}
	\item \texttt{FileDataLoader}: the primary function of this class is to load data from the input file and also to allow the user to save output data to the output file.
	\item \texttt{SharedPerformanceData}: this class stores data loaded from the input file into an array and also allows storage of output data into a separate array. But importantly it allows threads to access specific parts of the input data in a thread-safe manner. 
	\item \texttt{Worker}: this class contains functions to perform the important calculations, computations of deadlines missed and output data. This class defines the \texttt{operator ()} which encapsulates the main calculations, this class design is know as a \texttt{Functor}.
\end{enumerate}

The \texttt{C++} object-oriented class design can be visualised using a UML class diagram shown in figure \ref*{fig:mpbenchmark_UML_diagram}.

\begin{figure}[h] % Positioning preference: here, top, bottom, page
	\centering
	\includegraphics[width=1\textwidth, height=60cm]{~/Documents/Part_D_Modules/Individual_Project/Individual_report/figures/mpbenchmark_class.png} % Adjust the path and width as needed
	\caption{UML class diagram of the proposed \texttt{C++} solution.}
	\label{fig:mpbenchmark_UML_diagram} % Use this label to reference the figure
\end{figure}

These classes are used in the following sequence in the proposed solution[insert UML sequence diagram]:

The \texttt{mpbenchmark} implementation in the \texttt{C} language utilized the \texttt{printf} function for output throughout the application. Transitioning this functionality to \texttt{C++} posed a challenge, as \texttt{printf} relies on ``format string based" formatting, whereas \texttt{C++} typically uses ``stream-based" formatting through \texttt{std::cout}. This discrepancy was addressed in the \texttt{C++20} standard with the introduction of \texttt{std::format}. However, oddly enough, \texttt{std::format} was not supported by the \texttt{gcc} compiler version on the target system, despite it supporting the \texttt{C++20} standard\cite{std_format_gcc_compiler_version}. A workaround involved using the \texttt{fmt::print} function from the \texttt{fmt} library, which is noted as the inspiration behind \texttt{std::format}\cite{fmt_printing_library}. Detailed system specifications and a code snippet demonstrating the use of \texttt{fmt::print} are provided in the appendix.

The original \texttt{mpbenchmark} code included a command line argument allowing the user (or developer) to specify the engine type for computations. The application supports three distinct engine types. Importantly, the number of threads the application employs can be modified using the \texttt{taskset} command in \texttt{Linux}, which determines the CPU cores the application can run on. To improve this functionality, a second command line argument was added in the proposed solution. This new parameter enables users to set the number of threads the application should use. If this argument is omitted, the application automatically defaults to the maximum number of threads available on the system. This enhancement is detailed in a code snippet in the appendix.

To compile and link the application, the industry standard \texttt{CMake}\cite{cmake_about} software was used. In the \texttt{CMakeLists.txt}(the file used for building the project), the key aspects were specifying the \texttt{C++20} standard, including the \texttt{fmt} library and compilation with the \texttt{-O2} flag. This optimisation flag was also used by the authors of \texttt{mpbenchmark}\cite{mpbenchmark_paper} therefore it was used in the proposed solution for consistency. 

To further enhance the performance of the \texttt{mpbenchmark}, the \texttt{Valgrind/Callgrind} function profiler tool was deployed to find potential bottlenecks. \texttt{Callgrind} profiling results showed that part of the application where it approximates the value of $\pi$ had the highest self-cost. This code snippet is shown in the listing ~\ref{lst:pi_approximation_1} below:

\begin{lstlisting}[
	caption={Part of the code with the highest self-cost. It approximate $\pi$ using numerical integration.},
	label={lst:pi_approximation_1}
	]
// initialise variables for the pi calculation
const long num_steps = 1000000;
double step = 1.0 / static_cast<double>(num_steps);
double x{}, pi{}, sum{};

// performing numerical integration using the midpoint Riemann sum
for (int i = 0; i < num_steps; i++) {
	x = (i + 0.5) * step;
	sum += 4.0 / (1.0 + x * x);
}
pi = sum * step;
\end{lstlisting}

The problem with this part of the code is that it already part of the parallel regions where it is executed by each thread. It may seen like a good candidate for applying parallel \texttt{for loop} from the OpenMP library however creating nested threads beyond the number of threads available on the system does not always lead to higher performance and in many cases can degrade performance. Another way to improve performance is by using SIMD intrinsics, this a programming tool to improve single-threaded or sequential performance of a code it stands for ``Single Instruction Multiple Data". By using these intrinsics, programmers can write code that processes data in parallel directly within a single CPU cycle. An advantage of using \texttt{C++}(and/or \texttt{C}) is that SIMD intrinsics can be deployed whereas higher level languages like \texttt{Java} or \texttt{C\#} make it very difficult to access these. 

SIMD intrinsics vary by the target CPU, \texttt{x86} processors (which are found in most laptop and desktops) use \texttt{AVX2} instructions whereas \texttt{ARM} processors(commonly found in Apple products and embedded devices) use \texttt{NEON} instructions. In this project we need to use both as our proposed \texttt{C++} solution will be deployed on both desktop(\texttt{x86} processor) and Raspberry Pi devices(\texttt{ARM} based processor). The SIMD enhanced code can be summarised algorithmatically in the following steps and shown as a code snippet in listing ~\ref{lst:pi_approximation_2}: 

\begin{enumerate}
	\item Initialise \texttt{256-bit} wide vectors: each vector can hold four double-precision (\texttt{64-bit}) floating point numbers. The main initialisations would be a vector to hold four loop indices (\texttt{vec\_i}), a vector to calculate four values of $x$ (\texttt{vec\_x}), a vector to hold the result of the integrand (\texttt{vec\_temp}) and a vector to accumulate the sum (\texttt{vec\_sum}) after each loop iteration.
	\item \texttt{for loop} iterate until \texttt{num\_steps/4}:
	\begin{itemize}
		\item step 1: calculate the four midpoints $x$-values simultaneously using the vector \texttt{vec\_i} and hold result in \texttt{vec\_x}. Original formula used: \texttt{(i + 0.5) * step}.
		\item step 2: compute the value of the integrand in parallel using the four calculated $x$-values stored in \texttt{vec\_x}, store result in \texttt{vec\_temp}. Original formula used: \texttt{4 / (1 + x * x)}.
		\item step 3: accumulate the values from \texttt{vec\_temp} into the \texttt{vec\_sum} vector.
		\item step 4: increment loop indices vector \texttt{vec\_i} by \texttt{4}. 
	\end{itemize}
	\item Final summation: upon the completion of the loop, perform a horizontal sum on the vector (\texttt{vec\_sum}) that held the accumulated values.
	\item Calculation of $\pi$: sum is multiplied by the step size to approximate the value of $\pi$. Original formula used : \texttt{pi = sum * step}.
\end{enumerate}

The SIMD enhanced code implemented using \texttt{AVX2} instructions of listing ~\ref{lst:pi_approximation_1} is shown below in listing ~\ref{lst:pi_approximation_2}: 

\begin{lstlisting}[
	caption={SIMD enhanced code for approximation of $\pi$ using \texttt{AVX2} instructions. (Horizontal sum function in line 28 can be found in appendix).},
	label={lst:pi_approximation_2}
	]
double Worker::approximatePi(){
	double pi = 0.0; // Initialize pi to 0.0
	static constexpr long num_steps = 1000000; 
	static constexpr double step = 1.0 / static_cast<double>(num_steps); 
	
#if defined(__AVX2__)
	// Use AVX2 SIMD instructions if available
	double sum = 0; // Initialize scalar sum to accumulate final result
	
// Initialise all necessary 256-bit vectors
	__m256d vec_sum = _mm256_set1_pd(0.0);
	__m256d vec_step = _mm256_set1_pd(step);
	__m256d vec_half_step = _mm256_set1_pd(0.5 * step); 
	__m256d vec_one = _mm256_set1_pd(1.0); 
	__m256d vec_four = _mm256_set1_pd(4.0); 
	__m256d vec_x, vec_temp; 
	__m256d vec_i = _mm256_set_pd(3, 2, 1, 0); 
	__m256d vec_increment = _mm256_set1_pd(4); 

// Perform 4 computations at once, note "i" is incremented by 4 instead of 1
	for (int i = 0; i < num_steps; i += 4) {
		vec_x    = _mm256_add_pd(_mm256_mul_pd(vec_i, vec_step), vec_half_step); 
		vec_temp = _mm256_div_pd(vec_four, _mm256_add_pd(vec_one, _mm256_mul_pd(vec_x, vec_x))); 
		vec_sum  = _mm256_add_pd(vec_sum, vec_temp); 
		vec_i    = _mm256_add_pd(vec_i, vec_increment); 
	}
// Perform horizontal sum on vec_sum to get a scalar sum
	sum = hsum256_pd(vec_sum);
 // Multiply the sum by the step size to approximate pi
	pi = sum * step; 

#else
// If AVX2/SIMD instructions are unavailable then resort to using regular code... 
#endif

	return pi; 
}
\end{lstlisting}

As discussed, to utilize SIMD intrinsics on Raspberry Pi devices, \texttt{NEON} instructions must be employed. \texttt{NEON} instructions come with limitations, notably in their support for double precision floating points, which is restricted, and their vector width, which is only \texttt{128-bit}, compared to the \texttt{256-bit} vectors seen in \texttt{AVX2}\cite{neon_reference}. To accommodate \texttt{NEON}, two solutions were developed: one using single precision floating points (\texttt{float}) and the other using double precision floating points (\texttt{double}). The \texttt{NEON} code with \texttt{float} can perform four computations simultaneously, while the code with \texttt{double} can manage only two computations simultaneously, due to the \texttt{128-bit} vector's capacity to store four \texttt{float} values or two \texttt{double} values. Typically, \texttt{float} variables offer less decimal precision than \texttt{double} variables. These two SIMD-enhanced solutions, along with their decimal accuracy levels, will be compared in the results and discussion section. The \texttt{NEON} implementation follows the same algorithm as the code snippet shown in Listing~\ref{lst:pi_approximation_2}, and is available in the appendix.

In summary, for desktop (\texttt{x86}) processors, two main solutions have been developed: a novel \texttt{C++} solution and a SIMD-enhanced \texttt{C++} solution as discussed in Listing~\ref{lst:pi_approximation_2}. For the Raspberry Pi devices, three solutions have been created: the first is the novel \texttt{C++} solution, identical to that on the desktop, and the other two are the SIMD-enhanced versions utilizing \texttt{NEON} instructions with single and double precision floating point variables.

\section{Objective 2: \texttt{MobileNet}}
\section{Objective 3: \texttt{DeBaTE-FI platform}}

\chapter{Results and Discussion}
%Results and Discussion(8-10 pages)
\section{Objective 1: \texttt{mpbenchmark}}

Results from the proposed solution collected from desktop(\texttt{x86} processor) are shown below, they are presented in the legend ``C++" and ``C++(SIMD Optimised)". A benchmark plot along with a speedup plot are shown in figures ~\ref{fig:mpbenchmark_desktop_plot} and ~\ref{fig:mpbenchmark_desktop_speedup_plot} respectively. Full system specifications of the desktop(\texttt{x86}) along with the Raspberry Pi processors can be found in the appendix.

\begin{figure}[htbp] % Positioning preference: here, top, bottom, page
	\centering
	\includegraphics[width=1\textwidth, height=20cm]{~/Documents/Part_D_Modules/Individual_Project/Individual_report/figures/mpbenchmark_desktop.png} % Adjust the path and width as needed
	\caption{Mean benchmark plot of \texttt{mpbenchmark} collected from \texttt{x86} processor in seconds. The error bars represent 95\% confidence interval. (Lower is better).}
	\label{fig:mpbenchmark_desktop_plot} % Use this label to reference the figure
\end{figure}


\begin{figure}[htbp] % Positioning preference: here, top, bottom, page
	\centering
	\includegraphics[width=1\textwidth, height=20cm]{~/Documents/Part_D_Modules/Individual_Project/Individual_report/figures/mpbenchmark_desktop_speedup.png} % Adjust the path and width as needed
	\caption{Speedup plot collected from \texttt{x86} processor. The vertical black line shows the maximum physical cores of the processor. (Higher is better).}
	\label{fig:mpbenchmark_desktop_speedup_plot} % Use this label to reference the figure
\end{figure}

On desktop(\texttt{x86}) processor, \texttt{AVX2} instructions were used to implement SIMD intrinsics therefore we compare the decimal precision with the original unoptimised SIMD code, see table ~\ref{tab:c++_avx2_pi}.

\begin{table}[htbp]
	\centering
	\begin{tabular}{|c|c|c|}
		\hline
		\textbf{Programming language/configuration} & \textbf{Decimal value of $\pi$} & \textbf{Mean run time using maximum threads(s)} \\ \hline
		\texttt{C++}             & 3.1415926535897643 &  0.007659 \\ \hline
		\texttt{C++/AVX2}   & 3.1415926535899033 &  0.002173  \\ \hline
		\texttt{C}                 & 3.1415926535897643 & 0.011577 \\ \hline
		\texttt{Ada}             & 3.1415926535897643 &  0.008623\\ \hline
	\end{tabular}
	\label{tab:c++_avx2_pi}
	\caption{Comparing the decimal precision of the \texttt{AVX2} enhanced solution with the original.}
\end{table}

% Talk about C++ solution outperforming the original in both run times and speedup 
% C++ SIMD enhanced outperformed even more with a lower speedup.
% Talk about SMT's benefits if any
% Discuss AVX2's decimal precision

The proposed \texttt{C++} solution outperformed the original implementations in \texttt{C} and \texttt{Ada} by over 11\%, marking a significant reduction in runtime. Additionally, it achieved a higher speed-up compared to the \texttt{Ada} solution, as illustrated in figure~\ref{fig:mpbenchmark_desktop_speedup_plot}. The SIMD-enhanced solution achieved a remarkable runtime reduction of over 70\%, significantly outperforming all other solutions. However, this version did not show as large a speedup when run with an increased number of threads, which is not surprising given the already low runtime with a single thread. Moreover, the \texttt{AVX2}-enhanced code demonstrated decimal precision up to 12 decimal places, with discrepancies appearing from the 13th decimal place onwards, as shown in table~\ref{tab:c++_avx2_pi}. This suggests a slight consideration that using \texttt{AVX2} instructions might lead to reduced precision for calculations requiring high decimal accuracy. However, the reduction in decimal precision has been minor, and its effects on the application have been largely inconsequential. Given the dramatic improvement in performance with \texttt{AVX2} instructions, the minor loss of decimal precision seems negligible compared to the benefits. Both solutions have met their objectives, with the first achieving superior speed-up and the second providing insights into CPU performance when SIMD intrinsics are utilized.

Another noteworthy observation is the lack of performance gain when scaling from 8 to 16 cores. The processor used supports simultaneous multi-threading (SMT), commonly branded as ``Hyper-threading" for Intel CPUs, which is typical in modern \texttt{x86} processors. SMT theoretically allows each physical core to execute two threads, and an 8-core processor with SMT would appear to have 16 cores. However, the proposed solutions, along with other compiled languages like \texttt{C} and \texttt{Ada}, showed negligible performance gains by utilizing the virtual cores. \texttt{Java} experienced a slight performance degradation, while \texttt{C\#} was the only language to demonstrate a performance increase. Using the results from the \texttt{x86} processor, we can conclude that the additional cores provided by SMT did not enhance performance and, in some cases, even degraded it.

Results obtained from the latest Raspberry Pi 5 (\texttt{Cortex A-76} processor) are shown in figures ~\ref{fig:mpbenchmark_rpi5_plot} and ~\ref{fig:mpbenchmark_rpi5_speedup_plot}. The results obtained from the Raspberry Pi 4(\texttt{Cortex A-72} processor) are shown in figures ~\ref{fig:mpbenchmark_rpi4_plot} and ~\ref{fig:mpbenchmark_rpi4_speedup_plot}.

\begin{figure}[htbp] % Positioning preference: here, top, bottom, page
	\centering
	\includegraphics[width=1\textwidth, height=20cm]{~/Documents/Part_D_Modules/Individual_Project/Individual_report/figures/mpbenchmark_rpi5.png} % Adjust the path and width as needed
	\caption{Mean benchmark plot of results collected from Raspberry Pi 5(in seconds). The error bars represent 95\% confidence interval. (Lower is better).}
	\label{fig:mpbenchmark_rpi5_plot} % Use this label to reference the figure
\end{figure}

\begin{figure}[htbp] % Positioning preference: here, top, bottom, page
	\centering
	\includegraphics[width=1\textwidth, height=20cm]{~/Documents/Part_D_Modules/Individual_Project/Individual_report/figures/mpbenchmark_rpi5_speedup.png} % Adjust the path and width as needed
	\caption{Speedup plot collected from Raspberry Pi 5 processor. The vertical black line shows the maximum physical cores of the processor. (Higher is better).}
	\label{fig:mpbenchmark_rpi5_speedup_plot} % Use this label to reference the figure
\end{figure}

\begin{figure}[htbp] % Positioning preference: here, top, bottom, page
	\centering
	\includegraphics[width=1\textwidth, height=20cm]{~/Documents/Part_D_Modules/Individual_Project/Individual_report/figures/mpbenchmark_rpi4.png} % Adjust the path and width as needed
	\caption{Mean benchmark plot of results collected from Raspberry Pi 4(in seconds). The error bars represent 95\% confidence interval. (Lower is better).}
	\label{fig:mpbenchmark_rpi4_plot} % Use this label to reference the figure
\end{figure}

\begin{figure}[htbp] % Positioning preference: here, top, bottom, page
	\centering
	\includegraphics[width=1\textwidth, height=20cm]{~/Documents/Part_D_Modules/Individual_Project/Individual_report/figures/mpbenchmark_rpi4_speedup.png} % Adjust the path and width as needed
	\caption{Speedup plot collected from Raspberry Pi 4 processor. The vertical black line shows the maximum physical cores of the processor. (Higher is better).}
	\label{fig:mpbenchmark_rpi4_speedup_plot} % Use this label to reference the figure
\end{figure}

The decimal precision of using single and double precision floating point \texttt{NEON} instructions on the Raspberry Pi processors is compared in table ~\ref{tab:c++_neon_pi}.

\begin{table}[htbp]
	\centering
	\begin{tabular}{|c|c|c|c|}
		\hline
		\textbf{Programming language/configuration} & \textbf{Decimal value of $\pi$} & \textbf{Run time RPI5(s)} & \textbf{Run time RPI4(s)} \\ \hline
		\texttt{C++}                                                   & 3.1415926535897643 & 0.028356  & 0.115569 \\ \hline
		\texttt{C++/single-precision NEON}              & 3.141531467437744   &  0.016477 & 0.056352 \\ \hline
		\texttt{C++/double-precision NEON}             & 3.14159265358986     & 0.033399  & 0.116423 \\ \hline
		\texttt{C}                                                        & 3.1415926535897643 & 0.031107  & 0.116538\\ \hline 
		\texttt{Ada}                                                    & 3.1415926535897643  & 0.029549  & 0.117536 \\ \hline
	\end{tabular}
	\label{tab:c++_neon_pi}
	\caption{Comparing the decimal precision of the \texttt{NEON} enhanced solutions. RPI5 - Raspberry Pi 5, RPI4- Raspberry Pi 4. Mean benchmark time in seconds collected using maximum cores on the system(4 cores).}
\end{table}

% Talk about the proposed solution's performance in PI5 and PI4. 
% Talk about the speedup
% Talk about NEON solutions and their respective decimal precision.
% Can include the comparision with other solutions in the appendix, like Java and C# and Raspberry Pi 3 

The proposed solution outperformed both the \texttt{C} and \texttt{Ada} solutions in terms of runtime and speedup. On the Raspberry Pi 5, a slight reduction in runtime (about 4\%) and a notable improvement in speedup were observed. The results on the Raspberry Pi 4 were less impressive, showing only a 1\% improvement in performance and a marginal increase in speedup. Nevertheless, the proposed \texttt{C++} solution outperformed the original solutions in terms of runtime and speedup on both Raspberry Pi devices, although the improvement was marginal on the Raspberry Pi 4.

The \texttt{NEON} enhanced solutions using single precision floating-point produced over a 40\% reduction in time at the cost of lower speedup across threads and a reduced decimal precision to four decimal places. On the Raspberry Pi 5, the \texttt{NEON} solution using double precision failed to reduce runtime and produced the worst overall speedup across varying numbers of threads. On the Raspberry Pi 4, the \texttt{NEON} double precision solution did not reduce the runtime and produced a similar speedup to other solutions. However, the double precision solutions did offer far superior decimal precision, up to 12 decimal places. Given \texttt{NEON} instructions' reduced support for double precision floating points, developers must choose between single precision for significantly improved performance but lower decimal precision, and double precision, which did not improve performance on the tested embedded processors. These results are summarized in table~\ref{tab:c++_neon_pi}.

Benchmarks allow us to compare the latest Raspberry Pi 5 (\texttt{Cortex A-76}) against the older Raspberry Pi 4 (\texttt{Cortex A-72}). The \texttt{Cortex A-76} performed 75\% faster in terms of runtime when comparing the proposed \texttt{C++} solution, roughly \texttt{4x} faster. The \texttt{Cortex A-76} processor offered a 42\% reduction in runtime when \texttt{NEON(float)} enhanced code was used compared to a 51\% reduction in runtime for the \texttt{Cortex A-72}. Both processors exhibited a lower speedup when \texttt{NEON} instructions were used, similar to what was observed with the \texttt{x86} processor. The superior performance of the \texttt{Cortex A-76} may be attributed to a faster clock speed of 2.4 GHz compared to the 1.5GHz of the older \texttt{Cortex A-72} \cite{rasp_pi5_pi4_comparision}.

The proposed \texttt{C++} solution surpassed the previously developed \texttt{mpbenchmark}\cite{mpbenchmark_paper} on both \texttt{x86} and \texttt{ARM} processors. It offered better speedup across threads, and an alternate novel \texttt{SIMD} enhanced version (where the application decides whether to use \texttt{AVX2} or \texttt{NEON} depending on the processor type) can be utilized to analyse the CPUs' SIMD performance and scalability. This unprecedented improvement is part of an upcoming publication.

\section{Objective 2: \texttt{MobileNet}}
After parallelising \texttt{MobileNet} the results collected from desktop(\texttt{x86} processor) were compared with SIMD optimisations and without them. The benchmark and speedup plot are shown in figures ~\ref{fig:mobilenet_desktop_plot} and ~\ref{fig:mobilenet_desktop_speedup}. 

\begin{figure}[htbp] % Positioning preference: here, top, bottom, page
	\centering
	\includegraphics[width=1\textwidth, height=20cm]{~/Documents/Part_D_Modules/Individual_Project/Individual_report/figures/mobilenet_desktop.png} % Adjust the path and width as needed
	\caption{Mean benchmark plot of results collected from \texttt{x86} processor(in milliseconds). (Lower is better).}
	\label{fig:mobilenet_desktop_plot} % Use this label to reference the figure
\end{figure}

\begin{figure}[htbp] % Positioning preference: here, top, bottom, page
	\centering
	\includegraphics[width=1\textwidth, height=20cm]{~/Documents/Part_D_Modules/Individual_Project/Individual_report/figures/mobilenet_desktop_speedup.png} % Adjust the path and width as needed
	\caption{Speedup plot comparing the two solutions with results collected from \texttt{x86} processor. (Higher is better).}
	\label{fig:mobilenet_desktop_speedup} % Use this label to reference the figure
\end{figure}

The mean benchmark time of the original \texttt{MobileNet} application\cite{mobilenet_repo} was 1474.2 ms(using the \texttt{-O3} optimisation flag). The benchmark time using the maximum threads(16) with SIMD was 188.24ms and without SIMD it was 294.68ms. The SIMD optimisations are turned on and off as shown in listing  ~\ref{lst:mobilenet_parallel}. This presents a drastic reduction in run time of 87.2\% and 80.0\% for the SIMD and non-SIMD solutions respectively. Using multi-threading on the desktop processor yielded exceptional results and a dramatic improvement of the application. The results also indicate a performance degradation when using threads greater than 8 which is the maximum physical cores on the system. Both of the solutions produced a slight degradation of performance when going from 8 to 16 threads, suggesting once again the virtual cores from SMT did not yield in performance gains. 

The performance on the Raspberry Pi devices was a little more nuanced. The results of having different parallel regions and SIMD optimisations turned on or off were compared. The plots have the following legends:

\begin{enumerate}
	\item \texttt{3\_Parallel\_Regions\_SIMD}: all three functions \texttt{ConvLayer::forward}, \texttt{BatchNormalLayer::forward} and \texttt{ConvLayer::Addpad} have been parallelised with SIMD optimisations.
	\item \texttt{3\_Parallel\_Regions}: all three functions \texttt{ConvLayer::forward}, \texttt{BatchNormalLayer::forward} and \texttt{ConvLayer::Addpad} have been parallelised without SIMD optimisations.
	\item \texttt{2\_Parallel\_Regions\_SIMD}: only two functions \texttt{ConvLayer::forward} and \texttt{BatchNormalLayer::forward} have been parallelised with SIMD optimisations.
	\item \texttt{2\_Parallel\_Regions}: only two functions \texttt{ConvLayer::forward} and \texttt{BatchNormalLayer::forward} have been parallelised without SIMD optimisations.
\end{enumerate}

The results for Raspberry Pi 5 are shown in figures ~\ref{fig:mobilenet_rpi5_plot} and ~\ref{fig:mobilenet_rpi5_speedup}. The results for Raspberry Pi 4 are shown in figures ~\ref{fig:mobilenet_rpi4_plot} and ~\ref{fig:mobilenet_rpi4_speedup}. 

\begin{figure}[htbp] % Positioning preference: here, top, bottom, page
	\centering
	\includegraphics[width=1\textwidth, height=20cm]{~/Documents/Part_D_Modules/Individual_Project/Individual_report/figures/mobilenet_rpi5.png} % Adjust the path and width as needed
	\caption{Mean benchmark plot of results collected from \texttt{Cortex A-76} processor(in milliseconds). (Lower is better).}
	\label{fig:mobilenet_rpi5_plot} % Use this label to reference the figure
\end{figure}

\begin{figure}[htbp] % Positioning preference: here, top, bottom, page
	\centering
	\includegraphics[width=1\textwidth, height=20cm]{~/Documents/Part_D_Modules/Individual_Project/Individual_report/figures/mobilenet_rpi5_speedup.png} % Adjust the path and width as needed
	\caption{Speedup plot comparing different solutions with results collected from \texttt{Cortex A-76} processor. (Higher is better).}
	\label{fig:mobilenet_rpi5_speedup} % Use this label to reference the figure
\end{figure}

\begin{figure}[htbp] % Positioning preference: here, top, bottom, page
	\centering
	\includegraphics[width=1\textwidth, height=20cm]{~/Documents/Part_D_Modules/Individual_Project/Individual_report/figures/mobilenet_rpi4.png} % Adjust the path and width as needed
	\caption{Mean benchmark plot of results collected from \texttt{Cortex A-72} processor(in milliseconds). (Lower is better).}
	\label{fig:mobilenet_rpi4_plot} % Use this label to reference the figure
\end{figure}

\begin{figure}[htbp] % Positioning preference: here, top, bottom, page
	\centering
	\includegraphics[width=1\textwidth, height=20cm]{~/Documents/Part_D_Modules/Individual_Project/Individual_report/figures/mobilenet_rpi4_speedup.png} % Adjust the path and width as needed
	\caption{Speedup plot comparing different solutions with results collected from \texttt{Cortex A-72} processor. (Higher is better).}
	\label{fig:mobilenet_rpi4_speedup} % Use this label to reference the figure
\end{figure}


\chapter{Conclusion}
%Conclusion(2 pages)
% Your conclusion content here.
% commentary and interpretation of the main outcomes or findings, relative significance of findings, implications of results, limitations and future work.

% mpbenchmark: novel solution, C++ and SIMD suitable for both desktop and embedded processors to assess multi-core performance. State of the art acheivement, limitation would be how short the application is for more high-end CPUs. Future work could involve having a larger portion of application benchmark. SMT failed to deliver performance on x86 processor.

The first objective was met with strong results, surpassing the initial aim. A novel benchmark was developed using modern \texttt{C++}, which exceeded the capabilities of the previous \texttt{mpbenchmark}\cite{mpbenchmark_paper} in several aspects especially those implementations in \texttt{C} and \texttt{Ada}. The novel software design was modular and object-oriented, enhancing scalability. An alternate solution was also developed to assess CPU performance using SIMD intrinsics, with the application automatically selecting \texttt{AVX2} or \texttt{NEON} based on the CPU type. Results from desktop (\texttt{x86}) processors indicated negligible performance gains when utilizing SMT (or Intel's ``Hyper-threading"). The latest Raspberry Pi 5, equipped with the \texttt{Cortex A-76} processor, significantly outperformed the older Raspberry Pi 4's \texttt{Cortex A-72} by 75\%. On \texttt{ARM}-based CPUs, using single-point decimal precision with \texttt{NEON} instructions resulted in a 42-51\% performance gain but limited precision to four decimal places, posing a trade-off between performance and decimal precision.

Future research and development would focus on enhancing the benchmark to include more complex and demanding calculations to better assess the performance of high-end CPUs. The \texttt{x86} processor used in this project completed the benchmark application in less than 10 milliseconds, which might lead to a slightly misleading assessment of CPU capabilities if the application is too simplistic, despite utilising multiple threads. While this benchmark is effective for testing on embedded processors, it may be insufficient for assessing the performance of modern high-end CPUs, which now often feature more than 10 physical cores.

% MobileNet: novel solution and a dramatic performance boost. Desktop SMT failed to deliver again. On embedded devices need to tailor the solution to the target CPU. Limitation would be to improve the MobileNet to include the usage of modern C++ features. Future work would involve investigating SIMD optimisation on embedded/ARM CPUs for optimal performance. 
The second objective achieved novel results. The popular image classification algorithm \texttt{MobileNet}\cite{mobilenet_paper}\cite{mobilenet_repo} was parallelised using the \texttt{OpenMP} library. This optimised application provided a substantial 87\% reduction in runtime on the test \texttt{x86} processor when all system threads were utilised, though SMT did not yield any performance gains. On Raspberry Pi devices, performance improvements were more complex, and optimal performance was achieved when the application's multi-threaded architecture was specifically tailored to the target CPU. The optimised \texttt{MobileNet} application saw performance gains of 59.9\% and 65.2\% on Raspberry Pi 4 and 5, respectively, when all system threads were employed. However, the use of SIMD optimisations on the Raspberry Pi devices did not result in a significant performance improvement. In terms of application runtime, the latest Raspberry Pi 5 outperformed the Raspberry Pi 4 by 58\%. Furthermore, the optimal configuration on the Raspberry Pi 5 involved utilising three parallel regions of the application instead of only two on the Raspberry Pi 4, indicating that the \texttt{Cortex A-76} processor is better suited to exploit the \texttt{OpenMP} library's parallel regions.

Future work would focus on improvements to the \texttt{MobileNet} application’s software design. Enhancements could aim to increase modularity and leverage modern \texttt{C++} features. Another area for investigation is to determine why the \texttt{OpenMP} library's SIMD clause did not yield performance gains. This could be explored by manually employing \texttt{NEON} instructions and benchmarking the application to analyse the effects.

% DeBate-FI: unprecendented performance improvement in both local setup and the main setup. Limitation would redesigning the code of the application to make it more modular. Future work would be to collect more results from the main setup and use the C++ library instead of the Python obsolete telnet library. 
The third objective, the most challenging of all, was also met with strong results, achieving and even surpassing the initial aims. The \texttt{DeBaTE-FI} platform's application software, specifically its multiprocessing and multithreading capabilities, was optimised to reduce runtime and enhance scalability. The optimised solution successfully reduced runtime by 62.1\% in the local setup. In the main setup, the optimised solution achieved a 43.5\% reduction in runtime, confirming that the performance improvements are consistent beyond the local environment. Both results demonstrate significant and unprecedented improvements in performance. Additionally, an alternative solution was developed, which involved integrating an open-source \texttt{C++} library into the \texttt{Python} application for \texttt{telnet} communication. This solution also showed strong performance gains in the local setup, though not as substantial as the optimised \texttt{Python} solution, and it was not tested in the main setup due to project time constraints. However, the developed \texttt{C++} library can replace the existing, obsolete \texttt{Python} \texttt{telnetlib} library currently used by the \texttt{DeBaTE-FI} platform. Thus, this project not only enhanced the performance of the \texttt{DeBaTE-FI} platform but also offers a high-performance library that can be integrated into the \texttt{DeBaTE-FI} platform to replace its existing \texttt{telnet} library.

Future work would involve collecting more detailed results using the main setup to further analyse the optimized solution's performance. This task is time-consuming due to the long runtime of this application. Another area worth researching is the optimisation of the application's design, which currently remains convoluted and challenging to enhance, fix bugs, and potentially discover further optimisations for improved performance. The project also proposes using the developed \texttt{C++} library for \texttt{telnet} communication to replace the application's soon-to-be deprecated \texttt{telnetlib} \texttt{Python} library.


% Ensure bibliography starts on a new page and add it to the ToC.
\clearpage % Ensures bibliography starts on a new page.
\phantomsection % Ensures that the hyperlink anchor is correctly placed here.
\addcontentsline{toc}{chapter}{References}
\bibliographystyle{IEEEtran}
\bibliography{references} % Assumes your bibliography is in the 'references.bib' file.

% Ensure appendices start on a new page and add it to the ToC.
\newpage
\chapter{Appendices}
\begin{appendices}
	\renewcommand{\thesection}{\arabic{section}}
	\section{System and Raspberry Pi specifications}
\section{Objective 1: \texttt{mpbenchmark}}
\section{Objective 2: \texttt{MobileNet}}
\section{Objective 3: \texttt{DeBaTE-FI} platform}

\begin{lstlisting}[
	caption={Command line arguments of the new \texttt{mpbenchmark} solution.},
	label={lst:main_file_cpp}
	]
	int main(int argc, char *argv[]){
		int engine{};
		int numThreads{};
		
		if (argc > 1) {
			engine = std::atoi(argv[1]); // Convert the argument to an integer
		}
		
		if (argc > 2) {
			numThreads = std::atoi(argv[2]); // Convert the argument to an integer
		}
		
		// If the number of threads is not specified, default to the maximum available
		if (numThreads == 0) {
			numThreads = std::thread::hardware_concurrency();
		}	
		// ignore the other remaining code...
	}
\end{lstlisting}


\includepdf[
pages=-,
addtotoc={1, section, 1, DeBaTE-FI platform publication(maybe remove?), appendix:debate_fi},
pagecommand={\label{appendix:debate_fi}}
]{~/Documents/Part_D_Modules/Individual_Project/Individual_report/files/Debate_FI_platform.pdf}


  % Assuming this file contains your appendices.
\end{appendices}


\end{document}



