\subsection{Introduction}
The research topic focuses on parallel computing to optimize algorithms for both desktop and embedded CPUs. Parallel computing is widely regarded as a complex area within engineering and computer science. To develop or enhance multi-threaded solutions, a thorough understanding of the underlying algorithms and the specific goals of the application is crucial. This foundational knowledge is essential before attempting any modifications to ensure that changes do not deviate from the application’s original objectives. In our project, we concentrate on three applications: \texttt{mpbenchmark}, \texttt{MobileNet}, and \texttt{Debate-FI} platform. It is imperative to understand the background and functionality of these applications by utilising the existing publications centred around them. 

Another goal of the literature review is to identify existing solutions to avoid duplicating efforts. Adherence to programming best practices is also a critical aspect of the project, given the complexity of parallel computing. This project strives to ensure that all developed software solutions conform to the best practices and guidelines recommended by industry experts, necessitating a comprehensive literature review.

The literature review is structured to support the report's overall organization, with individual subsections dedicated to the three main objectives, the review covers providing background information, reviewing current research, and offering critical analysis. The review then explores build systems, focusing on how this project aims to develop cross-platform applications that can operate on different operating systems such as Linux, Windows, or macOS. Further research is directed towards software design and multi-threading in line with modern C++ guidelines, including discussions on relevant libraries, tools, and research on parallel algorithms. The conclusion section summarizes the findings from the preceding sections and underscores the significance of the reviewed literature to the research question.

\subsection{Objective 1: \texttt{mpbenchmark}}
This paper comes from a book that is published based on an international conference that takes place in Germany every year. The ARCS(Architecture of Computing Systems) conferences series has over 30 years of tradition reporting leading edge research in computer architecture and operating systems. The specific conference related to fist objective of this project is the one that took place in 2010. The book that is published after the conference contains a number of papers, our project focuses the paper titled JetBench: An Open Source Real-Time Multiprocessor Benchmark\cite{JetBench}. This is relevant as it contains a multiprocessor benchmark software that is suitable for our project objectives with the authors Muhammad Yasir Qadri, Dorian Matichard and Klaus D. McDonald Maier. 

The \texttt{JetBench} software emerged in response to the observed scarcity of real-time, multi-threaded benchmarks. Designed to simulate the thermodynamic calculations of jet engines in real time, \texttt{JetBench} serves as an application benchmark that not only simulates a realistic workload but also measures the time required to complete these calculations with different thread counts. Its primary objective is to assess the multi-core capabilities of CPUs. The workload generated by \texttt{JetBench} predominantly consists of Arithmetic Logic Unit (ALU) centric operations, including integer and double precision multiplication, addition, and division. These operations facilitate the computation of critical mathematical functions such as exponentiation, square roots, and conversions including the calculation of pi and degree-to-radian conversions. The software exhibits a significant parallel portion, constituting 88.6\% of its total operations. \texttt{JetBench} is developed in the C programming language and utilizes the \texttt{OpenMP} library to enable multi-threading, ensuring both efficiency and scalability. Furthermore, \texttt{JetBench} prioritizes portability, allowing it to assess the performance of a wide spectrum of systems, from low-end to high-end. This design philosophy ensures the avoidance of system-specific libraries or timers, underlining the software’s broad applicability and utility in performance evaluation across diverse computing environments.

The authors gathered their data using a system simulator powered by 16 \texttt{x86} CPUs, each running at a modest 20 MHz, under a Linux kernel environment. The rationale behind selecting such a low CPU clock speed remains undisclosed. Observations revealed that the application achieved optimal performance when operated with 8 threads, as indicated by the shortest runtime and minimal missed deadlines. Performance noticeably declined when the number of threads exceeded 8. Validation of these findings was further conducted on an Intel Dual Core processor, albeit with unspecified details, reinforcing the notion that performance for any threaded application diminishes beyond a certain threshold.

This research provides valuable insights into the foundational aspects of this application benchmark, though it is not without its limitations. For instance, the application encompasses a limited scope of computations, a decision made to ensure compatibility with lower-end CPUs. However, this constraint may render the application seemingly inadequate for testing on more advanced CPUs. Additionally, the omission of input/output operations was a deliberate choice to enhance the application's portability, albeit at the expense of being unable to assess the I/O capabilities of the target platform. Despite these limitations, the benchmark's strength lies in its portability and its efficacy in evaluating the multi-core performance across a spectrum of computing systems, from low to high-end. These attributes deem the application an apt selection for this project, which seeks to explore and develop multi-threading capabilities for both desktop and embedded platforms.

\subsection{Objective 2: \texttt{MobileNet}}
\subsection{Objective 3: \texttt{Debate-FI platform}}
\subsection{Cross platform build systems}
\subsection{\texttt{C++} guidelines and best practices}

\subsection{Literature review conclusion}
Summarize the main findings and debates covered in the review.
Reiterate the importance of the existing literature to your research question.
Outline how your research is positioned within the academic field based on your review.

